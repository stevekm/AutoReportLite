% ~~~~~~~~~~~~~~~~~~~~~~~
% 
% AutoReportLite analysis pipeline & reporting template
% by Stephen Kelly
% April 29, 2016
% https://github.com/stevekm/AutoReportLite
% 
% ~~~~~~~~~~~~~~~~~~~~~~~~
\documentclass[8pt]{beamer}\usepackage[]{graphicx}\usepackage[]{color}
%% maxwidth is the original width if it is less than linewidth
%% otherwise use linewidth (to make sure the graphics do not exceed the margin)
\makeatletter
\def\maxwidth{ %
  \ifdim\Gin@nat@width>\linewidth
    \linewidth
  \else
    \Gin@nat@width
  \fi
}
\makeatother

\definecolor{fgcolor}{rgb}{0.345, 0.345, 0.345}
\newcommand{\hlnum}[1]{\textcolor[rgb]{0.686,0.059,0.569}{#1}}%
\newcommand{\hlstr}[1]{\textcolor[rgb]{0.192,0.494,0.8}{#1}}%
\newcommand{\hlcom}[1]{\textcolor[rgb]{0.678,0.584,0.686}{\textit{#1}}}%
\newcommand{\hlopt}[1]{\textcolor[rgb]{0,0,0}{#1}}%
\newcommand{\hlstd}[1]{\textcolor[rgb]{0.345,0.345,0.345}{#1}}%
\newcommand{\hlkwa}[1]{\textcolor[rgb]{0.161,0.373,0.58}{\textbf{#1}}}%
\newcommand{\hlkwb}[1]{\textcolor[rgb]{0.69,0.353,0.396}{#1}}%
\newcommand{\hlkwc}[1]{\textcolor[rgb]{0.333,0.667,0.333}{#1}}%
\newcommand{\hlkwd}[1]{\textcolor[rgb]{0.737,0.353,0.396}{\textbf{#1}}}%

\usepackage{framed}
\makeatletter
\newenvironment{kframe}{%
 \def\at@end@of@kframe{}%
 \ifinner\ifhmode%
  \def\at@end@of@kframe{\end{minipage}}%
  \begin{minipage}{\columnwidth}%
 \fi\fi%
 \def\FrameCommand##1{\hskip\@totalleftmargin \hskip-\fboxsep
 \colorbox{shadecolor}{##1}\hskip-\fboxsep
     % There is no \\@totalrightmargin, so:
     \hskip-\linewidth \hskip-\@totalleftmargin \hskip\columnwidth}%
 \MakeFramed {\advance\hsize-\width
   \@totalleftmargin\z@ \linewidth\hsize
   \@setminipage}}%
 {\par\unskip\endMakeFramed%
 \at@end@of@kframe}
\makeatother

\definecolor{shadecolor}{rgb}{.97, .97, .97}
\definecolor{messagecolor}{rgb}{0, 0, 0}
\definecolor{warningcolor}{rgb}{1, 0, 1}
\definecolor{errorcolor}{rgb}{1, 0, 0}
\newenvironment{knitrout}{}{} % an empty environment to be redefined in TeX

\usepackage{alltt} % start LaTeX document
% set up parameters in R for use in the document

%
\begin{knitrout}\footnotesize
\definecolor{shadecolor}{rgb}{0.969, 0.969, 0.969}\color{fgcolor}\begin{kframe}


{\ttfamily\noindent\color{warningcolor}{\#\# Warning: running command 'readlink -f ../' had status 1}}\end{kframe}
\end{knitrout}




% ~~~~~~~~~~~~~~~~~~~~~~~~~~~~~~~~~~~~~~~~~~~~~~~~
% LaTeX settings start here:
\listfiles % get versions of files used for document compliaton, written at the end of the .log file for the report compilation!
\geometry{paperwidth=150mm,paperheight=105mm} % larger page size than normal for larger plots and more flexibility with font sizes
%\documentclass[8pt,xcolor={dvipsnames}]{beamer}
\setcounter{secnumdepth}{3} % how many levels deep before section headers stop getting numbers
\setcounter{tocdepth}{3} % table of contents depth
\usepackage{breakurl}
\usepackage{cite} % for citations, BibTeX I think
\usepackage{etoolbox} % this was not installed on HPCF, its in my home dir right now!! % has extra tools for LaTeX forloops, etc.; might not actually need this, use R loops to cat() LaTeX markup instead, much easier!
% \usepackage{forloop} % for LaTeX for loops; easier to use R loops to 'cat' TeX into the document instead!!
% \usepackage{tikz} % for custom graphics
%\usepackage{subcaption} %for subfigures%
% \usepackage{amsmath} % for math characters
\usepackage{graphicx} % good for advanced graphics options
\usepackage{tabularx} % for fancy table settings..
\usepackage{url} % for typesetting URLs, also file paths? 
\usepackage[multidot]{grffile} % support for image files with multiple '.' in the name
% \usepackage{adjustbox} % for auto-size box to put sample sheet into, also needs collectbox.sty
% \usepackage[usenames,dvipsnames]{color}
%%%%%%%%%%%%%experimental for xtable italics http://stackoverflow.com/questions/7996968/formatting-sweave-tables-two-challenges
% \usepackage{longtable} % allows for tables that break across pages
% \SweaveOpts{keep.source=TRUE}  % Keeps formatting of the R code.
%%%%%%%%%%%%%%%%%%%
%
% ~~~~~~ BEAMER SPECIFIC SETTINGS ~~~~~~~~ %
\makeatletter % turn on the '@' command character; needs to come before beamer settings
% \usetheme{Hannover} %  \usetheme{PaloAlto} % Bergen
% \usetheme[left]{Marburg} %  width= % hideothersubsections
\usetheme[left,hideothersubsections,width=3cm]{Marburg} %  width= % hideothersubsections
% location installed themes and such: /usr/share/texmf/tex/latex/beamer
\addtobeamertemplate{navigation symbols}{}{ % % this adds the page numbers at the bottom of the slide
    \usebeamerfont{footline}%
    \usebeamercolor[fg]{footline}%
    \hspace{1em}%
    \insertframenumber/\inserttotalframenumber
}
\makeatother % turn off the '@' command character; needs to come after beamer settings
% ~~~~~~~~~~~~~~~~~~~~~~~~~~~~~~~~~~~~~~~~~~~~%
% \graphicspath{/home/varitint/Dropbox/Lab/Teaching/Genomics_Class/Genomics_Lesson3_R!/With_embedded_R_code/figure/} % default path to find figures
%
%%%%%%%%%%
\IfFileExists{upquote.sty}{\usepackage{upquote}}{}
\begin{document}
% Create the Title page
\title[Sample Statistical Analysis]{SmithLab\_Taylor\_2016-12-31 Sample Statistical Analysis}
\author{Stephen Kelly}
\institute{\normalsize Dr. Aristotelis Tsirigos \\ PI: Dr. Smith \\ Genome Technology Center, \\ NYU Langone Medical Center, New York, NY 10016}
\date{\texttt{stephen.kelly@nyumc.org} \\ \today}
\titlegraphic{\includegraphics[width=0.25\textwidth]{figure/NYULMC_white}} % image to show on the title slide
\maketitle

% REPORT STARTS HERE!
%
\begin{kframe}


{\ttfamily\noindent\color{warningcolor}{\#\# Warning in file(file, "{}rt"{}): cannot open file '/sample-sheet.tsv': No such file or directory}}

{\ttfamily\noindent\bfseries\color{errorcolor}{\#\# Error in file(file, "{}rt"{}): cannot open the connection}}

{\ttfamily\noindent\bfseries\color{errorcolor}{\#\# Error in split(PipelineSampleSheet, (seq\_len(nrow(PipelineSampleSheet)) - : object 'PipelineSampleSheet' not found}}\end{kframe}\section{Sample Sheet}
\begin{kframe}

{\ttfamily\noindent\bfseries\color{errorcolor}{\#\# Error in eval(expr, envir, enclos): object 'tmpdf' not found}}\end{kframe}



%%%%%%%%%%%%%%%%%%%%%%%%%%%%
\section{Session Information}
% \begin{frame}{System and Session Information}
% \begin{frame}[fragile]{System and Session Information}
% \small{This report was prepared using the AutoReportLite template, available at \url{https://github.com/stevekm/AutoReportLite}}
\begin{knitrout}\footnotesize
\definecolor{shadecolor}{rgb}{0.969, 0.969, 0.969}\color{fgcolor}\begin{kframe}
\begin{alltt}
\hlkwd{system}\hlstd{(}\hlstr{'uname -srv'}\hlstd{,}\hlkwc{intern}\hlstd{=T)}
\end{alltt}
\begin{verbatim}
## [1] "Darwin 15.5.0 Darwin Kernel Version 15.5.0: Tue Apr 19 18:36:36 PDT 2016; root:xnu-3248.50.21~8/RELEASE_X86_64"
\end{verbatim}
\begin{alltt}
\hlkwd{sessionInfo}\hlstd{()}
\end{alltt}
\begin{verbatim}
## R version 3.3.0 (2016-05-03)
## Platform: x86_64-apple-darwin13.4.0 (64-bit)
## Running under: OS X 10.11.5 (El Capitan)
## 
## locale:
## [1] C
## 
## attached base packages:
## [1] stats     graphics  grDevices utils     datasets  methods   base     
## 
## other attached packages:
## [1] xtable_1.8-2    Hmisc_3.17-4    ggplot2_2.1.0   Formula_1.2-1  
## [5] survival_2.39-4 lattice_0.20-33 knitr_1.13     
## 
## loaded via a namespace (and not attached):
##  [1] Rcpp_0.12.5         cluster_2.0.4       magrittr_1.5       
##  [4] splines_3.3.0       munsell_0.4.3       colorspace_1.2-6   
##  [7] stringr_1.0.0       highr_0.6           plyr_1.8.4         
## [10] tools_3.3.0         nnet_7.3-12         grid_3.3.0         
## [13] data.table_1.9.6    gtable_0.2.0        latticeExtra_0.6-28
## [16] Matrix_1.2-6        gridExtra_2.2.1     RColorBrewer_1.1-2 
## [19] formatR_1.4         acepack_1.3-3.3     rpart_4.1-10       
## [22] evaluate_0.9        stringi_1.1.1       scales_0.4.0       
## [25] chron_2.3-47        foreign_0.8-66
\end{verbatim}
\begin{alltt}
\hlkwd{save.image}\hlstd{(}\hlkwc{compress} \hlstd{=} \hlnum{TRUE}\hlstd{)}
\end{alltt}
\end{kframe}
\end{knitrout}
\scriptsize{\LaTeX{} version: \LaTeXe~ \fmtversion}
% \end{frame}
\end{document}
